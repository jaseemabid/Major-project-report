\chapter{Platform Introduction}

One of the major challenges faced during the research and development
of this project is the number of tools that are required to start with
the project. Those tools are explained briefly here.

\section{Version Control Systems}

A Version control is a system that records changes to a file or set of
files over time so that you can recall specific versions later. It
allows you to revert files or the entire project back to a previous
state, compare changes over time, see who last modified something that
might be causing a problem etc. It means if you mess things up or lose
files, you can easily recover and you get all this for very little
overhead. A VCS is normally used for source code and
text files, but any type of file on a computer can be versioned. \\


The 2 major types of version control systems are,

\begin{enumerate}

\item{Centralized Version Control System (CVCS)}  \hfill \\
  \ldots
\item{Distributed Version Control System (DVCS)}  \hfill \\
  A DVCS keeps track of software revisions and allows many developers
  to work on a given project without necessarily being connected to a
  common network. These systems do not necessarily rely on a central
  server to store all the versions of a projects files. Instead, every
  developer clones a copy of a repository and has the full history of
  the project on their own hard drive. This copy (or clone) has all of
  the metadata of the original.

  The most popular DVCS are Git, Mercurial, Bazaar and also Arch,
  Monotone, Darcs etc.
\end{enumerate}
