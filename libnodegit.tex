\chapter{Libnodegit}

This section aims to give a detailed view into certain design concepts and
implementation details.

\section{Design decisions}

\begin{description}

\item[1. Implement only the absolute minimum in C++] \hfill \\
  Jumping a level down to C++ is definitely slow for small trivial tasks. For
  eg, if function expects \texttt{n} arguments, an exception is thrown at
  JavaScript layer before dropping down to C++ if it gets less than n arguments.
  If something can be done without C++, it is ignored.

\item[2. Polish API with JS wrapper functions] \hfill \\
  Almost all functions implemented in C++, end with a trailing underscore, like
  \texttt{reference\_} and is never exposed from the API. An equivalent wrapper
  function in JavaScript will be called \texttt{reference}, which will make sure
  that the arguments are correct, call the underlying C++ implementation and
  finally make sure that the result is returned in the right format. This rule
  is ignored for non trivial functions like head.

\item[3. Follow OOP standards as much as possible] \hfill \\
\item[4. Automate tasks with Makefiles and build tools like gyp] \hfill \\
\item[5. Write tests to ensure correctness wherever possible] \hfill \\

\end{description}

\section{API}
