\section{Git}

Git is fundamentally a content addressable filesystem with a VCS user
interface written on top of it. It is a simple key value data store
were you can insert any kind of content into it, and it will give you
back a key - a SHA1 id that you can use to retrieve the content again
at any time.

Git’s design was inspired by BitKeeper and Monotone\cite{inspiration}.
It was originally designed as a low-level version control system
engine on top of which others could write front ends. However, the
core project has since become a complete revision control system that
is usable directly. More of this will be discussed later.

\subsection{Characteristics \& Features}

Git’s design is a synthesis of Torvalds’s experience with Linux in
maintaining a large distributed development project, along with his
intimate knowledge of file system performance gained from the same
project and the urgent need to produce a working system

in short order. These influences led to the following implementation
choices:

\begin{description}

\item[1. Strong support for non-linear development] \hfill \\
  Git supports rapid branching and merging, and includes specific
  tools for visualizing and navigating a non-linear development
  history. Branches in git are very lightweight: A branch in git is
  only a reference to a single commit. With its parental commits, the
  full branch structure can be constructed.

\item[2. Distributed development] \hfill \\
  Like Darcs, BitKeeper, Mercurial, SVK, Bazaar and Monotone, Git
  gives each developer a local copy of the entire development history,
  and changes are copied from one such repository to another. These
  changes are imported as additional development branches, and can be
  merged in the same way as a locally developed branch.

\item[3. Compatibility with existing systems/protocols] \hfill \\
  Repositories can be published via HTTP, FTP, rsync, or a Git
  protocol over either a plain socket or ssh. Git also has a CVS
  server emulation, which enables the use of existing CVS clients and
  IDE plugins to access Git repositories. Subversion and SVK
  repositories can be used directly with git-svn.

\item[4. Efficient handling of large projects] \hfill \\
  Git is very fast and scalable, and performance tests showed it was
  an order of magnitude faster than some revision control systems, and
  fetching revision history from a locally stored repository can be
  one hundred times faster than fetching it from the remote server. In
  particular, Git does not get slower as the project history grows
  larger.

\item[5. Cryptographic authentication of history] \hfill \\
  The Git history is stored in such a way that the id of a particular
  revision (a commit in Git terms) depends upon the complete
  development history leading up to that commit. Once it is published,
  it is not possible to change the old versions without it being
  noticed.

\item[6. Toolkit-based design] \hfill \\
  Git was designed as a set of programs written in C, and a number of
  shell and perl scripts that provide wrappers around those programs.
  Although most of those scripts have since been rewritten in C for
  speed and portability, the design remains, and it is easy to chain
  the components together.

\item[7. Garbage accumulates unless collected] \hfill \\
  Aborting operations or backing out changes will leave useless
  dangling objects in the database. These are generally a small
  fraction of the continuously growing history of wanted objects. Git
  will automatically perform garbage collection when enough loose
  objects have been created in the repository or after a specific
  interval like a month or a week. Garbage collection can be called
  explicitly using ‘git gc –prune‘.

\item[8. Periodic object packing] \hfill \\
  Git stores each newly created object as a separate file. Although
  individually compressed, this takes a great deal of space and is
  inefficient. This is solved by the use of packs that store a large
  number of objects in a single file (or network byte stream) called
  packfile, delta-compressed among themselves. Newly created objects
  (newly added history) are still stored singly, and periodic
  repacking is required to maintain space efficiency.

\end{description}