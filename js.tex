\section{JavaScript}

JavaScript is most commonly used for client-side scripting in a browser, being
usedto manipulate Document Object Model (DOM) objects for example. The DOM is
not, however, typically provided by the JavaScript engine but instead by a
browser. The same is true of V8. Google Chrome provides the DOM. V8 does however
provide all the data types, operators, objects and functions specified in the
ECMA standard.

\subsection{The language}

\subsection{Browser wars and VM performance }

\subsection{Server side JavaScript}

\subsection{v8}

V8 is Google\’'s open source JavaScript engine written in C++ and is used in
Google Chrome, the open source browser from Google and implements ECMAScript as
specified in ECMA-262, 5th edition, and runs on a wide variety of platforms.

V8 can run standalone, or can be embedded into any C++ application. It compiles
and executes JavaScript source code, handles memory allocation for objects, and
garbage collects objects it no longer needs. V8’s stop-the-world, generational,
accurate garbage collector is one of the keys to V8’s performance. It enables
any C++ application to expose its own objects and functions to JavaScript code.
It’s up to the programmer to decide on the objects and functions he would like
to expose to JavaScript. This means an application can be written in C++, but an
API for that can be exposed in JavaScript. This is exactly what libnodegit is.
The core of the application is written in JavaScript and a richer API is
exposed.

\subsection{Node.js }

Node.js adds the missing pieces to v8. Node describes itself as \textit{evented
  I/O for V8 javascript}. Its a toolkit for writing extremely high performance
non-blocking event driven network servers in JavaScript. Think similar to
Twisted\cite{twisted} or EventMachine\cite{eventmachine} but for JavaScript
instead of Python or Ruby. On top of what v8 provides, Node adds non-blocking
file and network I/O, process management, and bunch of other handy things on top
of v8 and make it work on all platforms. Also, it gives access to numerous
libraries and a package management system called Node Package
Manager(npm)\cite{npm} to work with the library dependencies.

Its event-driven, non-blocking I/O model makes it lightweight and efficient,
perfect for data-intensive real time applications that run across distributed
devices. Node.js is a good platform to implement libnodegit because it is a I/O
driven application. Operations are performed on top of a text file based
repository in disk, and computation most of the time takes less time than disk
access. It is a task where Node.js can shine.
