\section{JavaScript}

JavaScript is a prototype-based scripting language that is dynamic, weakly
typed, and has first-class functions. Its syntax was influenced by the languages
C and Java and it also copies many names and naming conventions, but the two
languages are otherwise unrelated and have very different semantics. The key
design principles within JavaScript are taken from the Self and Scheme
programming languages. It is a multi-paradigm language, supporting
object-oriented, imperative, and functional programming styles.

JavaScript was originally implemented as part of web browsers so that
client-side scripts could interact with the user, control the browser,
communicate asynchronously, and alter the document content that was displayed.
Even though it was designed to work in a browser, its not restricted to browsers
or Document Object Model(DOM) manipulation. The DOM is not provided by the
JavaScript engine but instead by a browser and is not part of the language. It
is just part of the environment JavaScript is generally run.

JavaScript is one of the most popular computer programming languages in use
today.

\subsection{Server side JavaScript}

Netscape introduced an implementation of the language for server-side scripting
with Netscape Enterprise Server, first released in December, 1994 (soon after
releasing JavaScript for browsers). Since the mid-2000s, there has been a
proliferation of server-side JavaScript implementations. Node.js is one recent
notable example of server-side JavaScript being used in real-world applications.

The success of node ecosystem allowed to programmers to use the same language on
both client side and server side development using the numerous libraries and
led to its massive adoption. It powers a lot of web applications of considerable
size and is gaining users very fast now.

\subsection{v8}

V8 is Google\’'s open source JavaScript engine written in C++ and is used in
Google Chrome, the open source browser from Google and implements ECMAScript as
specified in ECMA-262, 5th edition, and runs on a wide variety of platforms.

V8 can run standalone, or can be embedded into any C++ application. It compiles
and executes JavaScript source code, handles memory allocation for objects, and
garbage collects objects it no longer needs. V8’s stop-the-world, generational,
accurate garbage collector is one of the keys to V8’s performance. It enables
any C++ application to expose its own objects and functions to JavaScript code
and It’s up to the programmer to decide on the objects and functions he would
like to expose to JavaScript. \textit{This means an application can be written
  in C++, but an API for that can be exposed in JavaScript}. This is exactly
what libnodegit is. The core of the application is written in JavaScript and a
richer API is exposed.

\subsection{Node.js }

Node.js adds the missing pieces to v8. Node describes itself as \textit{evented
  I/O for V8 javascript}. Its a toolkit for writing extremely high performance
non-blocking event driven network servers in JavaScript. Think similar to
Twisted\cite{twisted} or EventMachine\cite{eventmachine} but for JavaScript
instead of Python or Ruby. Node adds non-blocking file and network I/O, process
management, and bunch of other handy things on top of what v8 provides, and make
it work on all platforms. Also, it gives access to numerous libraries and a
package management system called Node Package Manager(npm)\cite{npm} to work
with the library dependencies.

Its event-driven, non-blocking I/O model makes it lightweight and efficient,
perfect for data-intensive real time applications that run across distributed
devices. Node.js is a good platform to implement libnodegit because it is a I/O
driven application. Operations are performed on top of a text file based
repository in disk, and computation most of the time takes less time than disk
access. It is a task where Node.js can shine.

\subsection{Miscellaneous tools}

\begin{description}

\item[1. npm]  \hfill \\
  NPM is a library and dependency manager for node.js applications

\item[2. node-gyp] \hfill \\
  Node-gyp is the build tool used. It can generate platform independent
  Makefiles from simpler configuration files.

\item[3. mocha] \hfill \\
  Libnodegit follows Behavior Driven Development(BDD) and mocha is the
  preferred testing framework.

\item[4. CoffeeScript] \hfill \\
  CoffeeScript is a tiny language that gets compiled into JavaScript and is used
  widely in the project.

\end{description}

Other than the development tools, reports are prepared with \LaTeX2e, diagrams
generated with \textit{graphviz} and \textit{Emacs} \textit{graphviz-mode}.
Documentation is written in \textit{markdown-mode} and powered with GitHub's
Gollum wiki.
