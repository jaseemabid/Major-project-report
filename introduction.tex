\chapter{Introduction}

Git\cite{git} is one of the most popular developer tools in use today and a lot
of very important opensource and proprietary projects use it to coordinate
development. Few of the major projects that use git are the \textit{Linux
  kernel}, Perl language, WebKit, Gnome, Apache Foundation, jQuery, Emacs and
git itself! Organizations like Facebook, Google and Microsoft use it internally
to develop their core products.

JavaScript\cite{javascript} is the only language available on the biggest
computing platform today - the web browser. It is available on almost all of the
modern devices and hence is one of the most popular languages today. Years of
development polished the language so well and it is very modern and powerful
now.

The two tools in question, Git and JavaScript are widely separated. Git was
originally designed to be used from a standard Linux shell and JavaScript
generally executes in the browser. Its not possible to shell out system commands
or write native extensions in C out of the box for JavaScript and It makes
linking both of them together a hard task. \textit{This project looks into the
  possibility of making them work together}.

\section{Overview}

Git is a free and open source distributed version control system designed to
handle everything from small to very large projects with speed and efficiency.
It was initially designed and developed by Linus Torvalds for Linux kernel
development; it has since been adopted by many other projects.

JavaScript (JS) is an interpreted computer programming language originally
implemented as part of web browsers so that client side scripts could interact
with the user, control the browser, communicate asynchronously, and alter the
document content that was displayed. It is prototype based, dynamic, weakly
typed, and has first class functions. Its syntax was influenced by the language
C. JavaScript copies many names and naming conventions from Java, but the two
languages are otherwise unrelated and have very different semantics. The key
design principles within JavaScript are taken from the Self and Scheme
programming languages. It is a multiparadigm language, supporting
object-oriented, imperative, and functional programming styles.

Git's primary user interface is the command line shell and it is really good in
its own way. There are alternate user interfaces for git, like
magit\cite{magit}, an interface heavily optimized for Emacs users. Alternate
interfaces expand the features of git core to new dimensions.

\emph{This project aims to build an alternate interface to Git, which can be
  used from JavaScript.}

\section{Why libnodegit?}

\newcommand\sectionTitle[1]{\begin{flushright}\textit{#1}\end{flushright}}
\sectionTitle{The motivations behind the project}

Amazing number of tools have been built using git, in ways that was never
expected. Git at its core is a content addressable file system which can be used
as a database. Its transport mechanisms can used to deploy applications,
synchronize data etc. The possibilities are endless and by providing an
alternate interface to git, libnodegit is extending the possibilities even more.

People have built amazing tools on top of git core, like issue
trackers\cite{gaskit}, blogging engines\cite{octopress}, project notification
systems\cite{git-dude} smarter FTP clients\cite{git-ftp} that transfer only the
changed files, production quality\cite{gollum-imporoved} wikis\cite{gollum} etc.

Git is the standard\cite{git-heroku} way to deploy apps on heroku\cite{heroku}
now, one of the largest PaaS\cite{PaaS} providers on the internet. Git is used
in a much simpler manner by designers\cite{git-designers}, with a smaller
subset. A lot of others tools and use cases are listed in the git
wiki\cite{git-tools} for reference.

By providing an API in a very popular, easy to use programming language,
libnodegit aims to help people build the next great tool, much easier.

\section{Implementation details}

At the heart of the project is a C++ core, written with libgit2\cite{libgit2}
and node.js\cite{node}. All communication from the JavaScript API, happens
through this core. Its not possible for JavaScript API to bypass the core and
talk to the git repository directly due to environment restrictions.

All interactions with git core are done with libgit2. It does the dirty work of
parsing the git object store and returning appropriate data structures back to
the application. Node.js allows this data to be returned back to the JavaScript
layer, by creating native bindings in C. It is one of the most important
component that makes the project possible.

JavaScript executes in various environments now, the most popular ones being web
browsers and now in a standard shell with \textit{node.js}. It will be possible
to write applications that can harness the power of git core, in a very
intuitive manner using libnodegit. The simple scripts written in JavaScript will
be executed by node.js.

\section{Background and recent work}

JavaScript is my primary programming language for last 3 years. I was introduced
to Git over 2 years back and I have worked extensively with it since then. I
spent a 3 month vacation working on a git subsystem called gitweb, trying to
rewrite the all JavaScript in the project in a modern and standards compliant
way, which was a partial success. Love for the tools that is involved and the
confidence that this can be achieved by me on time is the primary motivation
behind choosing this topic
