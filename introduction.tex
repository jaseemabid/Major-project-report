\chapter{Introduction}

Git is one of the most popular developer tools in use today and a lot of very
important opensource and closed source programs use it to coordinate
development. Few of the major projects that use git are the \textit{Linux
  kernel}, Perl language, WebKit, Gnome, Apache Foundation, jQuery, Emacs and
git itself! Organizations like Facebook, Google and Microsoft use it internally
to develop their core products. Its is of no question that properly versioning
the source code and keeping multiple backups can be a life saver in the case of
a system failure or security breach. It is an essential tool worth learning for
every programmer.

JavaScript comes with its own good and bad parts. The most important fact that
makes it stand out from every other programming language out there is that it is
\textit{the only available language in the biggest computing platform today -
  the web browser}. The lack of choice itself makes it important but years of
active development polished the language so well, that it is in a really good
shape today and is one of the most popular languages in use now.

Git and JavaScript are widely separated. Git is designed to be used from a
standard Linux shell and JavaScript generally executes in the browser. Its not
possible to shell out system commands or write native extensions in C out of the
box for JavaScript. It makes linking both of them together a hard task.
\textit{This project looks into the possibility of making them work together}.

\section{Overview}

Git is a free and open source distributed version control system designed to
handle everything from small to very large projects with speed and efficiency.
It was initially designed and developed by Linus Torvalds for Linux kernel
development; it has since been adopted by many other projects.

JavaScript (JS) is an interpreted computer programming language originally
implemented as part of web browsers so that client-side scripts could interact
with the user, control the browser, communicate asynchronously, and alter the
document content that was displayed. It is prototype-based, dynamic, weakly
typed, and has first-class functions. Its syntax was influenced by the language
C. JavaScript copies many names and naming conventions from Java, but the two
languages are otherwise unrelated and have very different semantics. The key
design principles within JavaScript are taken from the Self and Scheme
programming languages. It is a multi-paradigm language, supporting
object-oriented, imperative, and functional programming styles.

Git's primary user interface is the command line shell and it is really good in
its own way. There are alternate user interfaces for git, like
magit\cite{magit}, an interface heavily optimized for Emacs\cite{emacs} users.
Alternate interfaces expand the features of git core to new dimensions.

\textbf{This project aims to build an alternate interface to Git, which can be
  used from JavaScript.}

\section{Implementation details}

At the heart of the project is a C++ core, written with libgit2 and node.js. The
project core is similar to a \textit{Monolithic kernel}\cite{monolithic_kernel},
where all communication from the JavaScript API, happens through the kernel. Its
not possible for JavaScript API to bypass the core and talk to the git
repository directly like a \textit{Microkernel}\cite{microkernel}

All interactions with git core are done with libgit2. It does the dirty work of
parsing the git object store and returning appropriate data structures back to
the application core. Node.js allows this data to be returned back to the
JavaScript layer, by creating native bindings in C. It is this component that
makes the project possible.

JavaScript executes in various environments now, the most popular ones being web
browsers and \textit{node.js}. It will be possible to write applications that
can harness the power of git core, in a very intuitive manner using libnodegit.
The simple scripts written in JavaScript will be executed by node.js.

\section{Motivation}
